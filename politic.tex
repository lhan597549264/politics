\documentclass[11pt,a4paper,titlepage]{article}
\usepackage[a4paper]{geometry}
\usepackage[utf8]{inputenc}
\usepackage[english]{babel}
\usepackage{lipsum}

\usepackage{amsmath, amssymb, amsfonts, amsthm, fouriernc, mathtools}
% mathtools for: Aboxed (put box on last equation in align envirenment)
\usepackage{microtype} %improves the spacing between words and letters

\usepackage{graphicx}
\graphicspath{ {./pics/} {./eps/}}
\usepackage{epsfig}
\usepackage{epstopdf}


%%%%%%%%%%%%%%%%%%%%%%%%%%%%%%%%%%%%%%%%%%%%%%%%%%
%% COLOR DEFINITIONS
%%%%%%%%%%%%%%%%%%%%%%%%%%%%%%%%%%%%%%%%%%%%%%%%%%
\usepackage[svgnames]{xcolor} % Enabling mixing colors and color's call by 'svgnames'
%%%%%%%%%%%%%%%%%%%%%%%%%%%%%%%%%%%%%%%%%%%%%%%%%%
\definecolor{MyColor1}{rgb}{0.2,0.4,0.6} %mix personal color
\newcommand{\textb}{\color{Black} \usefont{OT1}{lmss}{m}{n}}
\newcommand{\blue}{\color{MyColor1} \usefont{OT1}{lmss}{m}{n}}
\newcommand{\blueb}{\color{MyColor1} \usefont{OT1}{lmss}{b}{n}}
\newcommand{\red}{\color{LightCoral} \usefont{OT1}{lmss}{m}{n}}
\newcommand{\green}{\color{Turquoise} \usefont{OT1}{lmss}{m}{n}}
%%%%%%%%%%%%%%%%%%%%%%%%%%%%%%%%%%%%%%%%%%%%%%%%%%




%%%%%%%%%%%%%%%%%%%%%%%%%%%%%%%%%%%%%%%%%%%%%%%%%%
%% FONTS AND COLORS
%%%%%%%%%%%%%%%%%%%%%%%%%%%%%%%%%%%%%%%%%%%%%%%%%%
%    SECTIONS
%%%%%%%%%%%%%%%%%%%%%%%%%%%%%%%%%%%%%%%%%%%%%%%%%%
\usepackage{titlesec}
\usepackage{sectsty}
%%%%%%%%%%%%%%%%%%%%%%%%
%set section/subsections HEADINGS font and color
\sectionfont{\color{MyColor1}}  % sets colour of sections
\subsectionfont{\color{MyColor1}}  % sets colour of sections

%set section enumerator to arabic number (see footnotes markings alternatives)
\renewcommand\thesection{\arabic{section}.} %define sections numbering
\renewcommand\thesubsection{\thesection\arabic{subsection}} %subsec.num.

%define new section style
\newcommand{\mysection}{
\titleformat{\section} [runin] {\usefont{OT1}{lmss}{b}{n}\color{MyColor1}} 
{\thesection} {3pt} {} } 

%%%%%%%%%%%%%%%%%%%%%%%%%%%%%%%%%%%%%%%%%%%%%%%%%%
%		CAPTIONS
%%%%%%%%%%%%%%%%%%%%%%%%%%%%%%%%%%%%%%%%%%%%%%%%%%
\usepackage{caption}
\usepackage{subcaption}
%%%%%%%%%%%%%%%%%%%%%%%%
\captionsetup[figure]{labelfont={color=Turquoise}}

%%%%%%%%%%%%%%%%%%%%%%%%%%%%%%%%%%%%%%%%%%%%%%%%%%
%		!!!EQUATION (ARRAY) --> USING ALIGN INSTEAD
%%%%%%%%%%%%%%%%%%%%%%%%%%%%%%%%%%%%%%%%%%%%%%%%%%
%using amsmath package to redefine eq. numeration (1.1, 1.2, ...) 
%%%%%%%%%%%%%%%%%%%%%%%%
\renewcommand{\theequation}{\thesection\arabic{equation}}

%set box background to grey in align environment 
\usepackage{etoolbox}% http://ctan.org/pkg/etoolbox
\makeatletter
\patchcmd{\@Aboxed}{\boxed{#1#2}}{\colorbox{black!15}{$#1#2$}}{}{}%
\patchcmd{\@boxed}{\boxed{#1#2}}{\colorbox{black!15}{$#1#2$}}{}{}%
\makeatother
%%%%%%%%%%%%%%%%%%%%%%%%%%%%%%%%%%%%%%%%%%%%%%%%%%




%%%%%%%%%%%%%%%%%%%%%%%%%%%%%%%%%%%%%%%%%%%%%%%%%%
%% DESIGN CIRCUITS
%%%%%%%%%%%%%%%%%%%%%%%%%%%%%%%%%%%%%%%%%%%%%%%%%%
\usepackage[siunitx, american, smartlabels, cute inductors, europeanvoltages]{circuitikz}
%%%%%%%%%%%%%%%%%%%%%%%%%%%%%%%%%%%%%%%%%%%%%%%%%%

\usepackage[UTF8]{ctex}

\makeatletter
\let\reftagform@=\tagform@
\def\tagform@#1{\maketag@@@{(\ignorespaces\textcolor{red}{#1}\unskip\@@italiccorr)}}
\renewcommand{\eqref}[1]{\textup{\reftagform@{\ref{#1}}}}
\makeatother
\usepackage[colorlinks,linkcolor=MyColor1]{hyperref}
\setlength{\parindent}{0pt}
%%%%%%%%%%%%%%%%%%%%%%%%%%%%%%%%%%%%%%%%%%%%%%%%%%
%% PREPARE TITLE
%%%%%%%%%%%%%%%%%%%%%%%%%%%%%%%%%%%%%%%%%%%%%%%%%%
\title{\blue Politics \\
\blueb summary\\
\textb written with \LaTeX }
\author{liuhan}
\date{\today}
%%%%%%%%%%%%%%%%%%%%%%%%%%%%%%%%%%%%%%%%%%%%%%%%%%



\begin{document}
\maketitle
\tableofcontents 
\newpage
\section{根本任务}
邓小平强调,社会主义的根本任务是发展生产力。\\
保持和发展党的先进性是马克思主义政党自身建设的根本任务和永恒课题\\
近代中国革命的根本任务是推翻帝国主义、封建主义和官僚资本主义的统治,从根本上推翻反动腐朽的政治上层
建筑,变革阻碍生产力发展的生产关系,为建设富强民主的国家、改善人民的生活、确立人民当家作主的政治制度扫清障碍,创造必要的前提\\
\section{主线}
新形势下全面提高党的建设科学化水平要牢牢把握加强党的执政能力建设,先进性和纯洁性建设这条主线\\
深化供给侧结构性改革是经济发展的主线\\
新时代加强和改进人民政协工作的总体要求:把服务实现"两个一百年"奋斗目标作为工作主线,
做好新时代党的民族工作,要把铸牢中华民族共同体意识作为党的民族工作的主线\\
服务民族复兴、促进人类进步,这是中国对外工作的一条主线\\
\section{根本目的}
增进民生福祉是发展的根本目的\\
供给侧结构性改革的根本目的是提高供给质量满足需要\\
推进全面依法治国,根本目的是依法保障人民权益\\
实现人民群众对美好生活的向往,是党和国家一切工作的根本目的\\
\section{首要任务}
全面贯彻实施完法是金面依法治国的首要任务\\
党的政治建设的首要任务∶保证全党服从中央.坚持党中央权威和集中统一领导\\
要把坚定理想信念作为为党的思想建设的首要任务\\
\section{抓手}
中国特色社会主义法治体系是推进全面依法治国的总抓手。\\
\section{方针}
无产阶级在同资产阶级建立统一战线时,必须坚持独立自主的原则,保持党在思想上、政治上和组织上的独立性,实行又联合又斗争的方针\\
在党与民主党派的关系上实行"长期共存、互相监督"的方针,在科学文化工作中实行"百花齐放、百家争鸣"的方针等\\
农业改造:积极领导、稳步前进的方针,循序渐进的步骤
手工业改造:积极领导、稳步前进的方针
一手抓物质文明.一手抓精神文明."两手抓,两手都要硬",这是我国社会主义现代化建设的一个根本方针
中国特色社会主义进入新时代,党的建设的根本方针是:坚持党要管党、全面从严治党
\section{三大}
"七大总结党的建设的历史经验.把党在长期奋斗中形成的优良作风概括为三大作风,即理论和实践相结合的作风,和人民群众紧密地联系在一起的作风,自我批评的作风。\\
毛泽东在《共产党人)发刊词》一文中.总结了中国革命两次胜利和两次失败的经验教训,揭示了中国革命发展的客观规律,指出∶"统一战线,武装斗争,党的建设,是中国共产党在中国革命中战胜敌人的三个法宝,三个主要的法宝。"正确地理解和处理了这三个问题及其相互关系,就等于正确地领导了全部中国革命。\\
针对抗日战争进入相持阶段后统一战线内部出现的危机,中国共产党提出的三大口号:巩固国内团结.反对内部分裂 力求全国进步,反对向后倒退 坚持抗战到底,反对中途妥协\\
\section{dang}
党的领导制度是我国的根本领导制度
勇于自我革命,从严管党治党,是我们党最鲜明的品格。
\section{根本原则}
要坚持中国共产党对人民军队的绝对领导.这是建设新型人民军队的根本原则\\
坚持党对国家安全工作的绝对领导,是做好国家安全工作的根本原则\\
\section{出发点落脚点}
以国家利益至上为准则,就是要把国家利益作为制定国家安全战略的出发点\\
为人民谋幸福、为民族谋复兴,是我们党的初心和使命,是党领导现代化建设的出发点和落脚点\\
人民幸福是国家富强、民族振兴的题中之义和必然要求,是国家富强、民族振兴的根本出发点和落脚点\\
要建设以全心全意为人民服务为唯一宗旨的人民军队。坚持全心全意为人民服务的宗旨,是建设新型人民军队的基本前提.也是人民军队一-切行动的根本准则和一-切工作的出发点与归宿。\\
\section{总目标}
明确全面推进依法治国总目标是建设中国特色社会主义法治体系、建设社会主义法治国家
明确全面深化改革总目标是完善和发展中国特色社会主义制度、推进国家治理体系和治理能力现代化\\
\section{根和魂}
党的领导制度明确了我国政治生活的领导关系、领导主体、领导对象,是中国特色社会主义制度体系的"根"和"源"
为人民谋幸福、为民族谋复兴,是我们党的初心和使命,也是新发展理念的"根"和"魂\\   
中华优秀传统文化是中华民族的"根"和"魂",是最深厚的文化软实力\\
\section{之义}
革命性是马克思主义的内在品质,是马克思主义的人民性、实践性和发展性的应有之义和必然要求\\
"没收官僚资本归新民主主义国家所有",是新民主主义革命的题中应有之义。\\
人民幸福是国家富强、民族振兴的题中之义和必然要求 \\
社会治理现代化是国家治理体系和治理能力现代化的题中应有之义\\
开放型世界经济的首要之义是反对保护主义\\
全面深化改革的总目标是实现社会主义现代化的题中应有之义\\
实"爱国者治港"原则(这是"一国两制"的应有之义,是香港特别行政区民主实践的本质要求)\\
\end{document}
